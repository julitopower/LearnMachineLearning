\documentclass[]{scrartcl}

% \documentclass[10pt]{scrartcl}

\renewcommand*\familydefault{\sfdefault}
\setkomafont{sectioning}{\bfseries}

\usepackage{framed}
\usepackage{xcolor}
\let\oldquote=\quote
\let\endoldquote=\endquote
\colorlet{shadecolor}{orange!15}
\renewenvironment{quote}{\begin{shaded*}\begin{oldquote}}{\end{oldquote}\end{shaded*}}

\definecolor{backcolour}{rgb}{0.95,0.95,0.92}

\usepackage{algorithm}
\usepackage{algorithmic}
\usepackage{listings}
\usepackage[]{palatino}
\usepackage{amssymb,amsmath}
\usepackage{ifxetex,ifluatex}
\usepackage{fixltx2e} % provides \textsubscript
\ifnum 0\ifxetex 1\fi\ifluatex 1\fi=0 % if pdftex
\usepackage[T1]{fontenc}
\usepackage[utf8]{inputenc}
\else % if luatex or xelatex
\ifxetex
\usepackage{mathspec}
\else
\usepackage{fontspec}
\fi
\defaultfontfeatures{Ligatures=TeX,Scale=MatchLowercase}
\fi
% use upquote if available, for straight quotes in verbatim environments
\IfFileExists{upquote.sty}{\usepackage{upquote}}{}
% use microtype if available
\IfFileExists{microtype.sty}{%
\usepackage{microtype}
\UseMicrotypeSet[protrusion]{basicmath} % disable protrusion for tt fonts
}{}
\usepackage[top=1cm, bottom=1cm, left=1cm, right=1cm]{geometry}
\usepackage[unicode=true]{hyperref}
\PassOptionsToPackage{usenames,dvipsnames}{color} % color is loaded by hyperref
\hypersetup{
colorlinks=true,
linkcolor=cyan,
citecolor=Blue,
urlcolor=cyan,
breaklinks=true}
\urlstyle{same}  % don't use monospace font for urls
\usepackage{graphicx,grffile}
\IfFileExists{parskip.sty}{%
\usepackage{parskip}
}{% else
\setlength{\parindent}{0pt}
\setlength{\parskip}{6pt plus 2pt minus 1pt}
}
\setlength{\emergencystretch}{3em}  % prevent overfull lines
\providecommand{\tightlist}{%
\setlength{\itemsep}{0pt}\setlength{\parskip}{0pt}}
\setcounter{secnumdepth}{5}
% Redefines (sub)paragraphs to behave more like sections
\ifx\paragraph\undefined\else
\let\oldparagraph\paragraph
\renewcommand{\paragraph}[1]{\oldparagraph{#1}\mbox{}}
\fi
\ifx\subparagraph\undefined\else
\let\oldsubparagraph\subparagraph
\renewcommand{\subparagraph}[1]{\oldsubparagraph{#1}\mbox{}}
\fi

\date{}

\begin{document}

{
\hypersetup{linkcolor=black}
\setcounter{tocdepth}{3}
\tableofcontents
}
\section{ML Refresher example
notebooks}\label{ml-refresher-example-notebooks}

These notebooks are inspired by the content of the fantastic book
``\href{https://www.amazon.com/Hands-Machine-Learning-Scikit-Learn-TensorFlow/dp/1492032646/ref=sr_1_3?crid=3SRL70QAD46FM}{Hands-On
Machine Learning with Scikit-Learn, Keras, and TensorFlow: Concepts,
Tools, and Techniques to Build Intelligent Systems 2nd Edition}'' by
\href{https://twitter.com/aureliengeron}{Aurélien Geron}. This book is a
goldmine that you should buy.

\section{Environment setup}\label{environment-setup}

We provide a conda environment configuration that defines all the
dependenies needed to run these example notebooks. To create a new conda
environment simply type:

\begin{verbatim}
conda env create -f=environment.yml 
\end{verbatim}

\subsection{Installing coda}\label{installing-coda}

\begin{verbatim}
wget https://repo.anaconda.com/miniconda/Miniconda3-latest-MacOSX-x86_64.sh -O ~/miniconda.sh
bash ~/miniconda.sh -b -p $HOME/miniconda
\end{verbatim}

Verify the installation by executing

\begin{verbatim}
conda env list
\end{verbatim}

\end{document}
